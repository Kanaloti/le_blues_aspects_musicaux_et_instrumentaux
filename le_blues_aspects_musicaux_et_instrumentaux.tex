% This file was converted to LaTeX by Writer2LaTeX ver. 1.0.2
% see http://writer2latex.sourceforge.net for more info
\documentclass[a4paper]{article}
\usepackage[utf8]{inputenc}
\usepackage[T1]{fontenc}
\usepackage[french]{babel}
\usepackage{textcomp}
\usepackage{amsmath}
\usepackage{amssymb,amsfonts,textcomp}
\usepackage{color}
\usepackage{array}
\usepackage{hhline}
\usepackage{hyperref}
\hypersetup{pdftex, colorlinks=true, linkcolor=blue, citecolor=blue, filecolor=blue, urlcolor=blue, pdftitle=, pdfauthor=, pdfsubject=, pdfkeywords=}
\usepackage[pdftex]{graphicx}
% Outline numbering
\setcounter{secnumdepth}{2}
\renewcommand\thesection{\Roman{section}}
\renewcommand\thesubsection{\arabic{subsection}}
% Page layout (geometry)
\setlength\voffset{-1in}
\setlength\hoffset{-1in}
\setlength\topmargin{2cm}
\setlength\oddsidemargin{2cm}
\setlength\textheight{25.7cm}
\setlength\textwidth{17.001cm}
\setlength\footskip{0.0cm}
\setlength\headheight{0cm}
\setlength\headsep{0cm}
% Footnote rule
\setlength{\skip\footins}{0.119cm}
\renewcommand\footnoterule{\vspace*{-0.018cm}\setlength\leftskip{0pt}\setlength\rightskip{0pt plus 1fil}\noindent\textcolor{black}{\rule{0.25\columnwidth}{0.018cm}}\vspace*{0.101cm}}
% Pages styles
\makeatletter
\newcommand\ps@Standard{
  \renewcommand\@oddhead{}
  \renewcommand\@evenhead{}
  \renewcommand\@oddfoot{}
  \renewcommand\@evenfoot{}
  \renewcommand\thepage{\arabic{page}}
}
\newcommand\ps@RightPage{
  \renewcommand\@oddhead{}
  \renewcommand\@evenhead{}
  \renewcommand\@oddfoot{}
  \renewcommand\@evenfoot{}
  \renewcommand\thepage{\arabic{page}}
}
\newcommand\ps@LeftPage{
  \renewcommand\@oddhead{}
  \renewcommand\@evenhead{}
  \renewcommand\@oddfoot{}
  \renewcommand\@evenfoot{}
  \renewcommand\thepage{\arabic{page}}
}
\makeatother
\pagestyle{Index}
\thispagestyle{Standard}
\title{BLUES}
\author{Kanaloti}
\date{2018-07-31}
\begin{document}

\clearpage\setcounter{page}{1}\pagestyle{LeftPage}
\thispagestyle{RightPage}
{\sffamily\bfseries {\maketitle}}
{\centering\sffamily\bfseries
Aspects musicaux et instrumentaux.
\par}




\setcounter{tocdepth}{10}
\renewcommand\contentsname{Table des mati\`eres}
\tableofcontents


 
\clearpage
\section{En vrac}
De Ed Friedland in \emph{Blues Bass}.
\begin{itemize}
\item Long I form : I I I I IV IV I I V IV I I
\item Quick IV form : A IV I I IV IV I I V IV I I. Selon \emph{Ed Friedland} in \emph{Blues Bass}, la long I peut-être utilisée pour la mélodie ou le chant, et la Quick IV pour les solos.
\item Parfois guitare rythmique et basse se rejoignent sur le turnaround quand elles ne jouent pas carrément la même chose. (Messin' with the Kid). C'est efficace.
\item Le pattern de boogie-woogie à la basse utilise toujours au moins 1 3 et 5 des accords I IV V. Mais la ligne de boogie est un pattern sur deux mesures qui doit être modifié quand on arrive mesure 9 et 10 du fait qu'il y a une seule mesure en V et une en IV.
\item Durant chaque blues gig (session d'enregistrements) un slow blues est joué. c'est l'occasion de contrebalancer les tempos medium et up et donnent une chance aux danseurs d'être "collés". Comme le tempo est lent chaque note que vous jouez est très exposée. Il est important de jouer une ligne qui ne soit pas out.
\item Quand vous jouez un slow blues attention au soliste. Il est typique pour un chanteur, ou instrumentiste, de diriger la dynamique du groupe pour suivre son feeling du moment. Il peut forcer l'intensité et soudainement la faire chuter pour créer un effet dramatique, c'est souvent le signal entre call et respons, et même de stopper complètement le tempo pour tenir une note aussi longtemps que possible, il signale aussi le retour au tempo avec un geste de la tête ou une vague de la main. Soyez sur que votre son reste consistant, et tenez vos notes à leur pleine durée. Faire des trous dans la ligne de basse peut créer un écart qui doit être discuté pour être effectif (et efficace).
\item Rhumba Blues : Le feel rhumba a trouvé sa place dans le blues à travers ses liens avec la Nouvelle-Orléans. Beaucoup des premières influences viennent de la Nouvelle-Orléans où les cultures africaines, européenne, caribéaines et américaines se sont mélangées, à l'image du \emph{Professor Longhair}.
\item Importance de l'intro. \emph{From the five}, \emph{from the two}, \emph{from the four} ou \emph{on the four}, \emph{wamp the one}, \emph{once trough}.
\item Intro from the five sur slow blues : one two three, batterie puis basse et guitare sur le temps 4.
\item les fins : elles sont très importantes mais il est rare qu'elles soient discutées avant le morceau (et vous êtes chanceux si l'intro est discutée).
\item Prenez le rythme de la kick sur le contretemps du temps 1 mesure 11. C'est un fait très commun dans beaucoup de fins. 
\item Il est commun pour un chanteurou un lead de couper au temps 1 de la mesure 10, après cela toutes les fins marche à partir de la mesure 11.
\end{itemize}

test test

\end{document}
